\documentclass[12pt,]{article}
\usepackage{lmodern}
\usepackage{amssymb,amsmath}
\usepackage{ifxetex,ifluatex}
\usepackage{fixltx2e} % provides \textsubscript
\ifnum 0\ifxetex 1\fi\ifluatex 1\fi=0 % if pdftex
  \usepackage[T1]{fontenc}
  \usepackage[utf8]{inputenc}
\else % if luatex or xelatex
  \ifxetex
    \usepackage{mathspec}
  \else
    \usepackage{fontspec}
  \fi
  \defaultfontfeatures{Ligatures=TeX,Scale=MatchLowercase}
    \setmainfont[]{Times New Roman}
\fi
% use upquote if available, for straight quotes in verbatim environments
\IfFileExists{upquote.sty}{\usepackage{upquote}}{}
% use microtype if available
\IfFileExists{microtype.sty}{%
\usepackage{microtype}
\UseMicrotypeSet[protrusion]{basicmath} % disable protrusion for tt fonts
}{}
\usepackage[margin=2.54cm]{geometry}
\usepackage{hyperref}
\hypersetup{unicode=true,
            pdftitle={Insert title of project here},
            pdfauthor={Rachel Landman},
            pdfborder={0 0 0},
            breaklinks=true}
\urlstyle{same}  % don't use monospace font for urls
\usepackage{longtable,booktabs}
\usepackage{graphicx,grffile}
\makeatletter
\def\maxwidth{\ifdim\Gin@nat@width>\linewidth\linewidth\else\Gin@nat@width\fi}
\def\maxheight{\ifdim\Gin@nat@height>\textheight\textheight\else\Gin@nat@height\fi}
\makeatother
% Scale images if necessary, so that they will not overflow the page
% margins by default, and it is still possible to overwrite the defaults
% using explicit options in \includegraphics[width, height, ...]{}
\setkeys{Gin}{width=\maxwidth,height=\maxheight,keepaspectratio}
\IfFileExists{parskip.sty}{%
\usepackage{parskip}
}{% else
\setlength{\parindent}{0pt}
\setlength{\parskip}{6pt plus 2pt minus 1pt}
}
\setlength{\emergencystretch}{3em}  % prevent overfull lines
\providecommand{\tightlist}{%
  \setlength{\itemsep}{0pt}\setlength{\parskip}{0pt}}
\setcounter{secnumdepth}{5}
% Redefines (sub)paragraphs to behave more like sections
\ifx\paragraph\undefined\else
\let\oldparagraph\paragraph
\renewcommand{\paragraph}[1]{\oldparagraph{#1}\mbox{}}
\fi
\ifx\subparagraph\undefined\else
\let\oldsubparagraph\subparagraph
\renewcommand{\subparagraph}[1]{\oldsubparagraph{#1}\mbox{}}
\fi

%%% Use protect on footnotes to avoid problems with footnotes in titles
\let\rmarkdownfootnote\footnote%
\def\footnote{\protect\rmarkdownfootnote}

%%% Change title format to be more compact
\usepackage{titling}

% Create subtitle command for use in maketitle
\providecommand{\subtitle}[1]{
  \posttitle{
    \begin{center}\large#1\end{center}
    }
}

\setlength{\droptitle}{-2em}

  \title{Insert title of project here}
    \pretitle{\vspace{\droptitle}\centering\huge}
  \posttitle{\par}
  \subtitle{\url{https://github.com/rml41/EDA_2020_Project.git}}
  \author{Rachel Landman}
    \preauthor{\centering\large\emph}
  \postauthor{\par}
    \date{}
    \predate{}\postdate{}
  

\begin{document}
\maketitle

\newpage
\tableofcontents 
\newpage
\listoftables 
\newpage
\listoffigures 
\newpage

\hypertarget{rationale-and-research-questions}{%
\section{Rationale and Research
Questions}\label{rationale-and-research-questions}}

Ellerbe Creek runs into the Falls Lake Resovoir, through the city of
Durham, North Carolina. Falls Lake serves as the source of drinking
water for the City of Raleigh and does not meet North Carolina standards
for \emph{chlorophyll a}, which is found in algae (SOURCE:
\url{https://durhamnc.gov/716/Falls-Lake}). Algal blooms generally come
from excess nutrients such as phosphorus and nitrogen. Ellerbe Creek is
one of the sources of excess nutrients and contaminents in Falls Lake.
The Ellerbe Creek Watershed has the highest population density of
Durham's watersheds, with an estimated 22\% impervious surface (SOURCE:
\url{https://files.nc.gov/ncdeq/Water\%20Quality/Planning/BPU/BPU/Neuse/Neuse\%20Plans/2009\%20Plan/Chapter\%201.pdf}).
It is impacted by both point and nonpoint sources and was found to
deliver the highest nutrient loads to Falls Lake (SOURCE:
\url{https://files.nc.gov/ncdeq/Water\%20Quality/Planning/BPU/BPU/Neuse/Neuse\%20Plans/2009\%20Plan/Chapter\%201.pdf}).
Ellerbe Creek and Falls Lake are both on the state's impaired water
bodies list (303(d) list) (SOURCE
\url{https://durhamnc.gov/711/Ellerbe-Creek-Watershed},
\url{https://www.usgs.gov/centers/sa-water/science/groundwatersurface-water-interaction-near-ellerbe-creek-durham-nc?qt-science_center_objects=0\#qt-science_center_objects}).
Ellerbe Creek was first listed on the 303(d) list in 1998 (SOURCE:
\url{https://files.nc.gov/ncdeq/Water\%20Quality/Planning/BPU/BPU/Neuse/Neuse\%20Plans/2009\%20Plan/Chapter\%201.pdf})

\newpage

\hypertarget{dataset-information}{%
\section{Dataset Information}\label{dataset-information}}

Nutrient data for this project were downloaded from the the Water
Quality Portal, a coorperative service sponsered by the United States
Geological Survey (USGS), the Environmental Protection Agency (EPA), and
the National Water Quality Monitoring Council (NWQMC) on February 27,
2020. Discharge data were downloaded for two stream gages along Ellerbe
Creek, HUC code 030202010403, from USGS using the data dataRetrieval
package in R. The dataset analyzed contains 21 monitoring locations with
measurments for nitrogen and phosphorus levels from 1982 to 2018 and
daily discharge data from 2008 to 20202. Not all locations had data for
each nutrient. Nitrogen and Phosphorus concentrations are recorded as
mg/L of Nitrogen or Phosphorus in various compounds including, nitrate,
nitrite, ammonia, ammonium, organic nitrogen, phosphate, and organic
phosphorus. The USGS gage locations are Club Blvd (0208675010),
upstream, and Gorman (02086849), downstream.

\begin{longtable}[]{@{}lllll@{}}
\toprule
Variable & Units & Range & Mean & Median\tabularnewline
\midrule
\endhead
Nitrogen & mg/L N & 0.37 - 33.00 & 7.18 & 2.82\tabularnewline
Phosphorus & mg/L P & 0.039 - 17.00 & 1.091 & 0.157\tabularnewline
Discarge Club & ft\^{}3/s & 0.20 - 781.00 & 9.39 & 1.28\tabularnewline
Discharge Gorman & ft\^{}3/s & 7.52 - 1750.00 & 48.84 &
20.50\tabularnewline
\bottomrule
\end{longtable}

\newpage

\hypertarget{exploratory-analysis}{%
\section{Exploratory Analysis}\label{exploratory-analysis}}

\hypertarget{discharge-data-wrangling}{%
\subsection{Discharge Data Wrangling}\label{discharge-data-wrangling}}

\hypertarget{nutrient-data-wrangling}{%
\subsection{Nutrient Data Wrangling}\label{nutrient-data-wrangling}}

\newpage

\hypertarget{analysis}{%
\section{Analysis}\label{analysis}}

\hypertarget{question-1-insert-specific-question-here-and-add-additional-subsections-for-additional-questions-below-if-needed}{%
\subsection{Question 1: \textless{}insert specific question here and add
additional subsections for additional questions below, if
needed\textgreater{}}\label{question-1-insert-specific-question-here-and-add-additional-subsections-for-additional-questions-below-if-needed}}

\hypertarget{question-2}{%
\subsection{Question 2:}\label{question-2}}

\newpage

\hypertarget{summary-and-conclusions}{%
\section{Summary and Conclusions}\label{summary-and-conclusions}}

\newpage

\hypertarget{references}{%
\section{References}\label{references}}

\textless{}add references here if relevant, otherwise delete this
section\textgreater{}


\end{document}
